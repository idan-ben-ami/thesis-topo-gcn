%
% You may wish to use some of the following options of the biuthesis
% package:
%
% fullpageDraft      avoid the margins necessary for proper binding and
%   just view or print a draft.
% beforeDefense      makes the personal acknowledgments invisible;
%   use this to print the copies you submit initially to the grad school
%   for sending to the opponent panel, i.e. thesis readers (who shouldn't
%   see those parts). For the final submission, after having successfully
%   defended - drop this option.
% noabbrevs          no notation & abbreviations list will be included
%   in the thesis.
%
% Additionally, you must specify the degree for which you're writing
% your thesis (MSc/PhD/MArch etc.)
%
\documentclass[MSc,noabbrevs]{misc/biuthesis}


% Definitions of info fields for the thesis - subject, advisor,
% faculty, acknowledgments, etc. etc. The thesis-fields file 
% contains Hebrew text, and should use the UTF-8 character set
% encoding (not iso-8859-8-i or windows codepage 1255).
% This file contains definitions of various fields used
% in various places throughout the thesis (in the title
% pages mostly). Whatever isn't define here has some
% default (and usually irrelevant) text.

\authorEnglish{Idan Ben-Ami}
\authorHebrew{עידן בן-עמי}

\titleEnglish{Graph Convolutional Networks \\ on \\ Static and Dynamic networks}
\titleHebrew{רשתות למידה על גרפים \\ המתבססות על \\ רשתות סטטיות ודינאמיות}

\disciplineEnglish{Computer Sciences}
\disciplineHebrew{מדעי המחשב}

\supervisionEnglish{
    This research was carried out under the supervision of 
    Prof.Yoram Louzoun, in the Faculty of Mathematics and 
    Prof.Yossi Keshset, in the Faculty of Computer Sciences.
}
\supervisionHebrew{המחקר בוצע בהנחייתו של פרופסור יורם לוזון, מהפקולטה למתמטיקה 
\\ ובהנחייתו של פרופסור יוסי קשת, מהפקולטה למדעי המחשב}

\GregorianDateEnglish{April 2020}
\GregorianDateHebrew{אפריל \textenglish{2020}}
\JewishDateEnglish{Nisan 5780}
\JewishDateHebrew{ניסן התש'פ}

\financialAcknowledgementEnglish{The generous financial help of Bar-Ilan University is gratefully acknowledged.}
\financialAcknowledgementHebrew{הכרת תודה מסורה לאוניברסיטת בר-אילן על מימון מחקר זה}

\publicationInfoEnglish{%
% The following may not be true regarding your own thesis...
% (The grad school guidelines now require that you mention the following regarding publications of your thesis work; but of course, remove this parenthesized note...; this is to be found in the \texttt{thesis-fields.tex} file. Note also that the document may need to be processed several times before the list of publications actually appears)

% Some results in this thesis have been published as articles by the author and research collaborators in conferences and journals during the course of the author's doctoral research period, the most up-to-date versions of which being:

% No need to specifically non-cite the items, all of the bib file's contents
% will appear here regardless
%\nociteacks{firstwork-foracks}
%\nociteacks{secondwork-foracks}
% \butcheredbibliography{pubinfo}{front/pubinfo}
}

\publicationInfoHebrew{%
להשלים עברית
\begin{otherlanguage}{english}%
% No need to mention the bibliography file this time, as it has already been used in
% the English invocation
\butcheredbibliography{pubinfo}{front/pubinfo}
\end{otherlanguage}%
}

\thesisbibfiles{back/general}
\thesisbibstyle{alpha}


% Personal acknowledgments (are only used for the post-exam version)
\personalAcknowledgementEnglish{
I would like to thank my advisor, my parents, my friends, etc. etc.

Add any thank-yous, acknowledgements, personal comments you wish to make here (in \texttt{personal-acks.tex}).

Note that this acknowledgements section only gets printed in the post-exam version of the thesis (i.e. if you leave out the \texttt{beforeDefense} option to your document class in \texttt{thesis.tex}.)
}

\personalAcknowledgementHebrew{

אני רוצה להודות למנחה שלי, להוריי, לחבריי, וכו' וכו'.

אפשר להוסיף עוד תודות והערות אישיות כאן.

שים/י לב: קטע זה של תודות מודפס בפועל רק בגרסת החיבור שלאחר-הבחינה (הווה אומר רק אם הסרת את האפשרות \textenglish{\texttt{beforeDefense}} מן האפשרויות המועברות ל-\textenglish{\texttt{document class}} בקובץ \textenglish{\texttt{thesis.tex}}.)
}


% A separate file for the abstract - in English and in Hebrew, so
% you must make sure it's also in the UTF-8 character set encoding.
%
% This file contains the abstract part of your thesis - in English and
% in Hebrew (within \abstractEnglish and \abstractHebrew respectively).
%
% Notes:
% - This file uses the UTF-8 character set encoding for the Hebrew
%   text not to get garbled. Keep it that way.
% - Assuming your thesis is mainly in English, Graduate School 
%   regulations mandate the following lengths for the abstracts:
%
%      Language    Min. Length   Max. Length
%     ---------------------------------------
%      English       200 words     500 words
%      Hebrew      1,000 words   2,000 words
%
%   so that the Hebrew abstract typically has some content from
%   the English introduction and an overview of the results, not
%   present in the English; it is not just a translation.

\abstractEnglish{
The main approaches for node classification in graphs are information propagation and the association of the class of the node with external information. State of the art methods merge these approaches through Graph Convolutional Networks.  We here use the association of topological features of the nodes with their class to predict this class.
Moreover, combining topological information with information propagation improves classification accuracy on the standard CiteSeer and Cora paper classification task. Topological features and information propagation produce results almost as good as text-based classification, without no textual or content information. We propose to represent the topology and information propagation through a GCN with the neighboring training node classification as an input and the current node classification as output. Such a formalism outperforms the state-of-the-art methods.

\paragraph
Graph Convolutional Networks (GCN) have emerged as one of the best methods to classify nodes using either external information or the graph structure itself through label propagation. However, current GCN approaches are based on a single snapshot of a graph or multi-graphs. We propose to combine GCN with recurrent neural networks and show that such a formalism outperforms the precision obtained in single graph-based predictions. We tested this method on a dataset of firms product similarity and shown it can be used to predict the future success of companies or forthcoming collapse. A similar comparison on a blog network and a prediction of the activity of bloggers produces similar results.
} % end of English abstract


\abstractHebrew{

Some hebrew

} % end of Hebrew abstract


% Just write down your abbervations here - or comment-out the command

\abbreviationsAndNotation{
\begin{tabular}{p{2cm}@{:\quad}l}
QED & quod erat demonstrandum (``what was to be shown'')\\
$c$ & the speed of light \\
$a\pm b$ & the closed interval $\left[a-b,a+b\right]$ \\
\end{tabular}
}


% Additional machinery relevant to any thesis
% (it's not part of the document class because it's not absolutely
% necessary and not everyone might like it)
\usepackage{misc/iitthesis-extra}

% Definitions useful for anything you write, which you also
% include in any articles, presentations, HW assignments and other
% documents. May contains macros for notation algebra, logic,
% calculus and other fields, as well as general shorthands and
% LaTeX tricks, and package use commands
% General-purpose definitions and inclusions
% you are using in any document 
% (regardless of its class and style files used),
% e.g. package uses:

%\usepackage{xspace}

% and macros/command defintions:

%\newcommand{\complexityclass}[1]{{\bf #1}\xspace}
%\newcommand{\NPTIME}{\complexityclass{NP}}

% For this template, we'll only have one single command,
% necessary for including graphics...
\usepackage{graphicx}% http://ctan.org/pkg/graphicx


% Definitions, settings and tweaks for this thesis specifically
% This file should contain your own definitions specific
% only to this thesis;

% What it contains below is used 
% simply for generating dummy text in the sample 
% content provided with the template (see mainchap1.tex);
% so you can safely delete this when creating your own
% thesis

\usepackage{lipsum}

% Another dummy text generator...
\def\qbfox#1{%
  \count1=0%
  \loop%
    \ifnum\count1<#1%
      \advance\count1 by 1%
      The quick brown fox jumped over the lazy dog. %
      \repeat%
  \relax%
}



\usepackage{pdfpages}

% If you are using WinEdt, and using a publication list on the the
% acknowledgments page, and are having problems getting your document
% to compile with the 'PDFLaTeXify' button, try uncommenting the
% following two lines;
% Also, you will need to PDFLaTeXify at least twice, as WinEdt misses
% an extra run. See also:
% http://tex.stackexchange.com/q/41727/5640
\usepackage{multibib}
\newcites{pubinfo}{Acknowledgement page references}
\def\iitthesisextramultibibdefs{}

\begin{document}

% Front Matter
% ------------

% The following command will typeset the outer cover page, the
% inner title page, the acknowledgments page, etc. - everything
% up to but not including the introduction
\makefrontmatter

% Main Matter
% ------------
%
\chapter{Introduction}
\label{chap:intro}

\subsection*{A classification problem}
Data classification is the mechanism of assigning a new observation to a category from a set of categories.
Suppose there is a truth classification function that takes new observation and outputs the true category. A classification problem is a process of finding a function (hypothesis) over all the potential functions (hypothesis space) that best fit the true classification function.
In those problems, there is a given set of examples (with or without their true outcome), which represents the behavior needed to be assessed. This set of examples is used to find the hypothesis which is as similar as possible to the true classification function.
In those problems, a classifier is a hypothesis that assigns a label to a given example.

\subsection*{Supervised learning}
There are various ways of finding this hypothesis (classifier), depending on the data and the existence of the expected outcome. Those are split into two main sub-groups, where each (or some) examples contain the expected outcome – those are the supervised or unsupervised learning mechanisms.
In the supervised learning approach, each example has the expected outcome (label). In the training process, those examples are used to build the best classifier that matches an example of the problem to the expected behavior.
In the unsupervised learning approach, the aim is to find similarities between subsets of examples and form groups of similar examples. This process is based solely on the data of the example and not considering its label (if exists).
There is another form of learning mechanism called semi-supervised learning in which the data is partially labeled. The behavior of the data (the classifier) is learned according to the labeled examples.

\subsection*{Graph classification problems}
A graph is an object $G=(V,E)$, where $V$ is a set of vertices and $E$ is a set of edges (i.e. a set of tuples of two vertices that may contain additional data – weight, direction, label, etc.)
The problem corresponds to data that is organized in a graph form, where the vertices are data entities and the edges represent relations between the vertices.
Only a subset of vertices are labeled, therefore to predict the labels of the rest of the unlabeled vertices, behavioral analysis of this labeling pattern is required, i.e. a semi-supervised classification problem.

\subsection*{Previous work}
One way to perform this behavioral analysis is to analyze the data in each node by using classical approaches.
Although, this approach ignores the graph’s structure which is the relation between the vertices.

% <!-- Talk about Kipf’s approach -->
To consider the graph's structure, one has proposed a neural-network-based model called GCN (Graph Convolutional Network). Each layer of the network is a function of the product of the graph's adjacency matrix with the input vector to the current layer and weights optimized by the network. This combination brings the topological aspects of the graph into consideration by picking the attributes of each node according to each node's neighborhood. The given problem tested on the model is a classification of textual articles into subjects where the input data to the model is a Bag-Of-Words (BOW) of each article. The adjacency matrix is first symmetrized and normalized, hence the graph's structure is transformed into an undirected graph, meaning losing a lot of structural information.

\subsection*{This research}
The GCN model described above is partly considering the graph's structure and adding extra crucial information into account. Though, it does not consider the global and local aspects of those relations, nor does it considering the asymmetric aspect of the data - i.e. when the graphs are directed.
To solve the asymmetric issue of directed graphs, an upgrade to the model was proposed. This upgrade added extra care in each layer to handle both the adjacency matrix and the transpose to the adjacency matrix, which handles both directions of the graph's edges.
In some cases, collecting the data can be a very expensive and error-prone process which may lead to bad prediction results.
To solve this, another approach of prediction was proposed relying solely on topological information - various, local and global, topological features were extracted from the graph itself.

In all the models, represented until this point, the data was organized in a graph structure that is taken in a specific window of time. Hence, the behavior across all the vertices (labels of the vertices) was represented by a single hypothesis.
Though, things get more difficult when the behavior of the vertices changes in time. Given a series of snapshots of data, represented by a series of graphs with the same vertices but with different connectivity, a better classifier can be built using this change of behavior in time.
The first basic approach to do so is to consider each snapshot as a static graph (as discussed previously) and train a classifier for each snapshot by itself. As this is a solution to the problem, it contains the earlier disadvantage of not considering historical data in the training process of each time window.
Another approach would be to train a single model on all the graphs and evaluate it on a portion of the vertices on each snapshot by itself. This approach saves training time and data and can better handle the dynamic aspects of the data. Although, it produces worse results than the previously suggested solution, as it does not take into account the order of the snapshots, i.e. the order of time.
A better solution would be to build a dynamic, sequential model (RNN) that is fed serially by the snapshots in time. Each layer is fed with the current snapshot and previously saved data, so the training occurs both on current behavior and historical data.

% <!-- Another issue is that it can't deduct efficiently on newly arrived data, because it doesn't study the trend in time. Should talk about size of data & models? (probably not) -->

% \chapter{Topological based classification of paper domains
using graph convolutional networks}
\label{chap:topo_gcn}

\includepdf[pages=-]{gcn-topo-tex/topo_gcn}

\chapter{Combined GCN LSTM for vertex classification}
\label{chap:nips}

\includepdf[pages=-]{gcn-lstm-tex/nips_2019}

%
% and then cover:
% - The methods used in the research
% - The research results
% - Discussion and conclusions from the results
%
% but not necessarily with a specific chapter for each of them.
%
% Then you have your main chapters (although these might still
% include an initial chapter on technical preliminaries, experimental
% system setup, and/or a final chapter with summary, discussion and further
% research direction or questions)

% \include{main/prelims}
% \include{main/mainchap1}
\chapter{Conclusion and future work}
\label{chap:conclusion}

Our work is divided into several parts.
Firstly, we have introduced a new input method to the original GCN model which improved its performance and have presented zero-internal-information classification.
In this input method, we have calculated several features based on the topological structure of the graph. Among them was the counting of n'th order of the neighbor's labels of each vertex.
When using the external features alone, we managed to reach close results to the results achieved when using the internal features. Moreover, we managed to surpass them when mixed the internal and external features.
In some cases, the inner-information is very expensive or not valid at all, therefore this experiment is highly valuable.
Another extension that we deployed to the basic model was considering the direction aspect of the graph. In the original model, the directivity of the graph was eliminated by making the adjacency matrix symmetrical. Whereas we allowed asymmetric adjacency matrix by extending the inner-layer handling of the data, which improved the results in deeper networks.
Lastly, we have introduced a new model to handle sequences of graphs which represents snapshots in time of dynamic data.
This new dynamic model was superior to the static model which trained and tested on each snapshot individually. Furthermore, its superiority was tested on two different networks which changed in time.

Our approach relies strongly on the connectivity of the graph. hence, if the graph is too sparse or too dense, the quality of the classifications may decrease.
Our model is very costly with resources due to the inner layer calculations and the size and amount of resources restrict the size of networks that can be handled. It is mainly affecting the LSTM based model's training.

\section{Future work}
Relevant future work may include handling weighted graphs, experimenting deeper networks, optimizing the various models' parameters, extend the methodologies to reduce the graph sizes to deploy our models on larger datasets, extending the LSTM dynamic model and asses our model on a wider range of datasets.

%
% Add any appendices here; they must come _before_ the bibliography
%
\appendix
%\noappendicestocpagenum
%\addappheadtotoc
% \include{main/appendix1}

% Back Matter
% ------------

% The following command will typeset the bibliography,
% then typeset the Hebrew part of the thesis:
% - Cover page
% - Title page
% - Acknowledgements page
%  (NO table of contents or list of figures in Hebrew)
% - (Extended) abstract (1000-2000 words)
%
% based on information you've provided in the thesis-fields file
% (including the relative paths to your bib files). The Hebrew
% content will be typeset in _reverse_page_order_, i.e. first
% in the file will be the last page of the abstract, and the
% Hebrew cover page will be the last page of the file.
%
\makebackmatter
% \makehebrewmatter

% The resulting PDF can be printed and taken straight to binding,
% i.e. you do not need to flip any pages anywhere. Of course,
% mind the LaTeX error and warning messages, overfull hboxes etc.

\end{document}
